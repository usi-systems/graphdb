\section{Problem Description}

The problem is to determine a partitioning of the disk blocks into sub-blocks
that would minimize query I/O. Of course, partitioning into sub-blocks results
in an increase in storage cost. The storage cost must stay below some
specified limit.

There are two possibilities: \emph{overlapping} and \emph{non-overlapping}
partitioning. In the overlapping case, attributes can be replicated in the
sub-blocks. The non-overlapping case is a true partition.

We start by defining the cost model, which formalizes the query I/O and storage
costs.

\subsection{Cost Model}
Let $Q$ be the query workload, where each query $q\in Q$ accesses a set of
attributes $q.A$ and traverses parts of the graph for the time range
$q.T=[q.t_s,q.t_e]$. We denote the set of all attributes as $A$. Given a block
$B$, we denote its time range as $B.T$, which is the union of the time ranges
of its temporal neighborlists. Let $s(a)$ denote the size of an attribute $a$.
We use $c_n(B)$ to denote the number of temporal neighborlists within block
$B$ and $c_e(B)$ to denote the total number of edges in the temporal
neighborlists within the block. We overload the notation for block size and
use $s(B)$ to denote the size of a block $B$. We have: 
\begin{equation}
s(B) = c_e(B) \cdot \Big(16 + \sum_{a\in B.A} s(a)\Big) + c_n(B) \cdot 12  
\end{equation}
Here, $16$ corresponds to the cost of storing the edge id and the timestamp,
and $12$ corresponds to the cost of storing the head vertex ($8$ bytes) plus
the number of entries ($4$ bytes) for a temporal neighborlist. 

Our goal is to create a potentially overlapping partitioning of a block,
denoted by $\mathcal{P}(B)$. We call the partitions \emph{sub-blocks}. We have
$\bigcup_{B'\in \mathcal{P}(B)} B'.A = A$. We aim to find the function
$\mathcal{P}$ that minimizes the query I/O over $B$, while keeping the
relative storage overhead beyond a limit, say $1+\alpha$ times the original.
If we represent the query I/O as $L(\mathcal{P}, B)$ and the relative storage
overhead as $H(\mathcal{P}, B)$, our goal is to find:
\begin{equation}
\mathcal{P} \leftarrow \mbox{argmin}_{\{\mathcal{P}: H(\mathcal{P}(B)) < \alpha\}} L(\mathcal{P},B)
\end{equation}

The storage overhead can be formalized as follows for the non-overlapping case:
\begin{equation}
H(\mathcal{P}, B) = (|\mathcal{P}(B)|-1)\cdot\Big(1-\frac{c_e(B)\cdot \sum_{a\in A} s(a)}{s(B)}\Big) 
\end{equation}

For the general case (including overlapping), we can formulate the storage
overhead as follows:
\begin{equation}
H(\mathcal{P}, B) = \frac{\sum_{B'\in \mathcal{P}(B)} s(B')}{s(B)} - 1 
\end{equation}

%TODO: note about why we have two different cases

\subsubsection{Query I/O}

Let $m$ be a function that maps a query $q$ to the set of sub-blocks that are
accessed to satisfy it for a relevant block $B$ under a given partitioning
$\mathcal{P}$. For the case of non-overlapping attributes, we have:
\begin{equation}
m(\mathcal{P}, B, q) = \{B': B'\in \mathcal{P}(B) \wedge q.A \cap B'.A \ne \emptyset\}  
\end{equation}

For the case of overlapping attributes, we use a simple heuristic to define the
set of sub-blocks to be used for answering the query.
Algorithm~\ref{alg:greedyM} captures it. The idea is to select sub-blocks that
bring the highest relative marginal gain, which is defined as the size of
attribute data that contributes to the query result, relative to the sub-block
size. While computing the marginal gain, attributes that are covered by
sub-blocks that are already selected are not considered.
%
\begin{algorithm}[ht]
\scriptsize
\caption{m-overlapping($\mathcal{P}, B, q$)}
\label{alg:greedyM}
\KwData{$\mathcal{P}$: partitioning function, $B$: block, $q$: query}
$S\leftarrow \emptyset; R\leftarrow \emptyset$ \tcp*{Selected attributes; Resulting sub-blocks}
\While(\tcp*[f]{While unselected attributes remain}){$S \subset q.A$}{
  $B' \leftarrow \mbox{argmax}_{B'\in\mathcal{P}(B)\setminus R} \sum_{a\in B'.A \cap q.A \setminus S} \frac{c_e(B') \cdot s(a)}{s(B')}$
  $S \leftarrow S \cup B'.A$\tcp*{Extend the selected attributes}
  $R\leftarrow R \cup B'$\tcp*{Extend the selected sub-blocks}
}
\Return R \tcp*{Final set of sub-blocks covering the query attributes}
\end{algorithm} 
%
The query I/O cost for a block is then given by:
\begin{equation}
L(\mathcal{P}, B) = \sum_{q\in Q} \mathbf{1}(q.T \cap B.T \neq \emptyset) \cdot \sum_{B'\in m(\mathcal{P}, B, q)} \!\!s(B')
\end{equation}

%\paragraph{\textbf{Other useful equations}}
%\noindent The following are unused, but may be needed for the partitioning
%algorithm.
%
%Let $f: \mathbb{R}\times A \rightarrow \mathbb{N}$ be a function that maps a
%attribute-timestamp pair $(a, t)$ to the frequency of queries that access
%attribute $a$ for time $t$. More formally:
%\begin{equation}
%f(a,t) = \sum_{q\in Q} \textbf{1}(a\in q.A \wedge t\in q.T)
%\end{equation}
%
%Let $T$ represent a time range $[T.s, T.e]$. We denote the frequency of
%attribute access for the time interval $T$ for attribute $a$ as follows:
%\begin{equation}
%f(a, T) = \int_{T.s}^{T.e} f(a, t)\cdot dt =  \sum_{q\in Q} |q.T \cap T| 
%\end{equation}


