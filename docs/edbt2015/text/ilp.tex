\section{ILP Solution}\label{sec:ilp}
\noindent
In this section, we formulate the optimal railway design problem as a mixed
Integer Linear Program (ILP). The main challenge is to represent the objective
function and the constraint as a linear combination of potentially integer
variables.

For the ILP formulation, we define a number of binary ($0$ or $1$) variables: 
\begin{itemize}
\item $x_{a,p}$: $1$ if attribute $a$ is in partition $p$, $0$ otherwise.
\item $y_{p,q}$: $1$ if partition $p$ is used by query $q$, $0$ otherwise.
\item $z_{a,p,q}$: $1$ if partition $p$ is used by query $q$ and attribute $a$
is in partition $p$, $0$ otherwise.
\item $u_{p}$: $1$ if partition $p$ is assigned at least $1$ attribute, $0$ otherwise.
\end{itemize}

Each of these variables serve a purpose:
\begin{itemize}
\item $x$s define the attribute-to-partition assignments.
\item $y$s help formulate the query I/O contribution of each partition due to
the graph structure they contain (excluding their assigned attributes).
\item $z$s help formulate the query I/O contribution of each partition, only
considering the attributes they are assigned.
\item $u$s help formulate the storage overhead requirement.
\end{itemize}

In total, we have $|A|\cdot(|A|+1)\cdot(|Q|+1)$ variables. Here, we assume that
the maximum number of partitions is fixed. In fact, we cannot have more
partitions than attributes, so the number of partitions is upper bounded by
$k=|A|$, and thus $0\leq p<k$. However, some of these partitions can be empty
in the optimal solution, which means that the number of partitions found by
the ILP solution is typically lower than the maximum possible. A simple
post-processing step removes empty partitions and creates the final
partitioning to be used. 

Finally, we define a helper notation for representing whether a variable is
accessed by a query or not: $q(a)\equiv \mathbf{1}(a \in q.A)$. 

We are now ready to state the ILP formulation. We separate the cases of
non-overlapping and overlapping partitioning, as the former case can be
formulated using smaller number of constraints.

\subsection{Non-Overlapping Partitions}\label{subsubsec:nov-ilp}
\noindent
We start with the objective function, that is the total query I/O, which is to
be minimized:
\begin{eqnarray}
\sum_{q\in Q} w(q) \cdot \Big(\sum_{p=1}^{k} \!\!&&\!\! (16\cdot c_e(B) + 12\cdot c_n(B))\cdot
y_{p,q}\nonumber\\ 
&+& \sum_{a\in A} s(a)\cdot c_e(B)\cdot z_{a,p,q}\Big)\label{eq:no-obj}
\end{eqnarray}

In Eq.~\ref{eq:no-obj}, we simply sum for each query and each partition, and
add the I/O cost of reading in the structural information found in a
sub-block, if the partition is used by the query. We then sum over each
attribute as well, and add the I/O cost of reading in the attributes. Note
that $z_{a,p,q}$ could have been replaced with $x_{a,p}\cdot y_{p,q}$, but
that would have made the objective function non-linear. 

We are now ready to state our constraints. Our first constraint is that, each
attribute must be assigned to a single partition. Formally:
\begin{eqnarray}
\forall_{a\in A}, \sum_{p=1}^{k} x_{a,p} = 1
\end{eqnarray}

Our second constraint is that, if a query $q$ contains an attribute $a$
assigned to a partition $p$, then partition $p$ is used by the query, i.e.,
$y_{p,q}=1$. In essence, we want to state: $\forall_{\{p,q\}\in [1..k]\times
Q}, y_{p,q} = \mathbf{1}(\sum_{a\in A} q(a)\cdot x_{a,p}>0)$. 

In order to formulate this constraint, we use the following ILP  construction:
Assume we have two variables, $\beta_1$ and $\beta_2$, where $\beta_2\in[0,1]$
and $\beta_1\geq 0$. We want to implement the following constraint: $\beta_2 =
\mathbf{1}(\beta_1 > 0)$. This could be expressed as a linear constraint as
follows, where $K$ is a large constant guaranteed to be larger than $\beta_1$
for all practical purposes:
\begin{eqnarray}
&& \beta_1 - \beta_2 \geq 0\nonumber\\
&& K\cdot\beta_2 - \beta_1 \geq 0\label{eq:beta-ilp}
\end{eqnarray}

We now apply this construction to our second constraint, where
$\beta_1=\sum_{a\in A} q(a)\cdot x_{a,p}$ and $\beta_2=y_{p,q}$. This results
in the following linear constraints:
\begin{eqnarray}
\forall_{\{p,q\}\in [1..k]\times Q}, 
    &&  \sum_{a\in A} q(a)\cdot x_{a,p} - y_{p,q} \geq 0 \nonumber\\
\forall_{\{p,q\}\in [1..k]\times Q}, 
    &&  K\cdot y_{p,q} - \sum_{a\in A} q(a)\cdot x_{a,p}  \geq 0 
\end{eqnarray}

Our third constraint is that, if an attribute $a$ is assigned to a partition
$p$, and partition $p$ is used by a query $q$, then the corresponding $z$
variable must be set to $1$. That is, we want: $\forall_{\{a,p,q\}\in A\times
[1..k]\times Q}, z_{a,p,q}=\mathbf{1}(x_{a,p} = y_{p,q} = 1)$. We express this
as a linear  constraint, as follows:
\begin{eqnarray}
\forall_{\{a,p,q\}\in A\times [1..k]\times Q},
    && z_{a,p,q} - (x_{a,p} + y_{p,q}) \geq -1\label{eq:no-z}
\end{eqnarray}

In Eq.~\ref{eq:no-z}, when the $x$ and $y$ variables are both $1$, the  $z$
variable is simply forced to be $1$. Otherwise, the $z$ variable can be either
$0$ or $1$, but since the $z$ variables appear in the objective function as
positive terms, the solver will set them to $0$ to minimize the I/O cost (Note
that the $z$ variables do not appear in any other constraint). 

Our fourth constraint is that, if a partition is non-empty, then its
corresponding $u$ variable must be set to $0$. In other words,  we want
$\forall_{p\in[1..k]}, u_p = \mathbf{1}(\sum_{a\in A} x_{a,p}>0)$. This is
expressed as linear constraints, as follows:
\begin{eqnarray}
\forall_{p\in[1..k]},
    && \sum_{a\in A} x_{a,p} - u_p \geq 0 \nonumber\\
\forall_{p\in[1..k]},
    && K\cdot u_p - \sum_{a\in A} x_{a,p} \geq 0 \label{eq:no-u}
\end{eqnarray}

Eq.~\ref{eq:no-u} uses the same construction as the second constraint, where
$\beta_1=\sum_{a\in A} x_{a,p}$ and $\beta_2=u_p$.

Our fifth, and the last, constraint deals with the storage overhead. We want to
 make sure that the storage overhead does not go over $\alpha$. Recall that
for  the non-overlapping attributes case, the storage overhead depends on the
number of partitions used (Eq.~\ref{eq:nov-ohead}). That means that the only 
ILP variables it depends on are the $u$s. In particular, the number of
partitions used is given by $\sum_{p=1}^{k} u_p$. This results in the
following linear constraint:
\begin{equation}
\sum_{p=1}^{k} u_p \leq 1 + \frac{\alpha}
  {1-\frac{c_e(B)\cdot \sum_{a\in A} s(a)}{s(B)}}
\end{equation}

\begin{figure}[!t]
%\begin{mdframed}
\begin{eqnarray}
\text{minimize}  
    \sum_{q\in Q} w(q)\cdot \Big(\sum_{p=1}^{k} \!\!&&\!\! (16\cdot c_e(B) + 12\cdot c_n(B))\cdot y_{p,q}\nonumber\\
    &+& \sum_{a\in A} s(a)\cdot c_e(B)\cdot z_{a,p,q} \Big) \nonumber\\
\text{subject to}&&\nonumber\\
\forall_{a\in A}, 
    && \sum_{p=1}^{k} x_{a,p} = 1\nonumber\\
\forall_{\{p,q\}\in [1..k]\times Q}, 
    &&  \sum_{a\in A} q(a)\cdot x_{a,p} - y_{p,q} \geq 0 \nonumber\\
\forall_{\{p,q\}\in [1..k]\times Q}, 
    &&  K\cdot y_{p,q} - \sum_{a\in A} q(a)\cdot x_{a,p}  \geq 0 \nonumber\\
\forall_{\{a,p,q\}\in A\times [1..k]\times Q},
    && z_{a,p,q} - (x_{a,p} + y_{p,q}) \geq -1\nonumber\\
\forall_{p\in[1..k]},
    && \sum_{a\in A} x_{a,p} - u_p \geq 0 \nonumber\\
\forall_{p\in[1..k]},
    && K\cdot u_p - \sum_{a\in A} x_{a,p} \geq 0 \nonumber\\    
&& \sum_{p=1}^{k} u_p \leq 1 + \frac{\alpha}
  {1-\frac{c_e(B)\cdot \sum_{a\in A} s(a)}{s(B)}} \nonumber
\end{eqnarray}
%\end{mdframed}
\caption{ILP formulation for the non-overlapping optimal railway design.}
\label{fig:nov-ilp}
\end{figure}

The final ILP formulation for the non-overlapping partitioning is given in
Figure~\ref{fig:nov-ilp}. We have a total of $|A|^2\cdot|Q| +
2\cdot|A|\cdot|Q| + 3\cdot|A| + 1$ constraints and the objective function
contains $|A|\cdot|Q|\cdot(1+|A|)$ variables.

\subsection{Overlapping Partitions}\label{subsubsec:ov-ilp}
\noindent
We present an ILP formulation of the problem, as we did for the case of
non-overlapping partitions in Section~\ref{subsubsec:nov-ilp}. We use the same
set of variables and the same objective function. However, the formulation of
the constraints differ. 

Our first constraint is that, each attribute must be assigned to at least one
partition. Formally:
\begin{eqnarray}
\forall_{a\in A}, \sum_{p=1}^{k} x_{a,p} \geq 1
\end{eqnarray}

As our second constraint, we require that for each attribute contained in a
query, there needs to be a partition that is used by that query and that
contains the attribute in question. Formally:
\begin{eqnarray}
\forall_{\{a,q\}\in A\times Q}, \sum_{p=1}^{k} z_{a,p,q} \geq q(a) 
\end{eqnarray}

\begin{sloppypar}
As our third constraint, we require that if a query is using an attribute from
a partition, then that partition must contain the attribute. I.e., we need
to link the $z$ variables with the $x$ variables as
$\forall_{\{a,p,q\}\in A\times [1..k]\times Q}, (z_{a,p,q} = 1) \implies 
(x_{a,p} = 1)$. This can be stated as linear constraints:
\begin{eqnarray}
\forall_{\{a,p,q\}\in A\times [1..k]\times Q}, x_{a,p} - z_{a,p,q} \geq 0 
\end{eqnarray}
\end{sloppypar}

As our fourth constraint, we require that if a query is using at least one
attribute from a partition, then that partition must be used by the query.
I.e., we need to link the $z$ variables with the $y$ variables as
$\forall_{\{p,q\}\in [1..k]\times Q}, y_{p,q} = \mathbf{1}(\sum_{a\in A}
z_{a,p,q}>0)$. As before, we use the ILP construction from
Eq.~\ref{eq:beta-ilp} for this, where $\beta_2=y_{p,q}$ and $\beta_1 =
\sum_{a\in A} z_{a,p,q}$. We get:
\begin{eqnarray}
\forall_{\{p,q\}\in [1..k]\times Q}, 
    &&  \sum_{a\in A} z_{a,p,q} - y_{p,q} \geq 0 \nonumber\\
\forall_{\{p,q\}\in [1..k]\times Q}, 
    &&  K\cdot y_{p,q} - \sum_{a\in A} z_{a,p,q} \geq 0 
\end{eqnarray}

Our fifth constraint is that, if an attribute $a$ is assigned to a partition
$p$, and partition $p$ is used by a query $q$, then the corresponding
$z_{a,p,q}$ variable must be set to $1$. This is same as the formulation for
the non-overlapping case from Eq.~\ref{eq:no-z}.

Our sixth constraint is that, if a partition is non-empty, then its
corresponding $u$ variable must be set to $0$. Again, this is same as the
formulation for the non-overlapping case from Eq.~\ref{eq:no-u}.

Our seventh, and the last, constraint deals with the storage overhead. However,
the storage overhead formulation for the overlapping case is different from 
the one for the non-overlapping case. This is because the overhead does not
merely depend on the number of partitions, as attributes might have to be read
multiple times from different partitions (due to the overlaps). As a result,
we express the overhead using base variables as in the objective function.
Formally:
\begin{eqnarray}
\sum_{p=1}^{k} \Big((16\cdot c_e(B) &+& 12 \cdot c_n(B)) \cdot u_p  \nonumber \\ 
+ \sum_{a\in A} s(a) \!\!&\cdot&\!\! c_e(B)\cdot x_{a,p} \Big) \leq s(B)\cdot (1+\alpha)
\end{eqnarray}

\begin{figure}[!t]
%\begin{mdframed}
\begin{eqnarray}
\text{minimize}  
    \sum_{q\in Q} w(q)\cdot \Big(\sum_{p=1}^{k} \!\!&&\!\! (16\cdot c_e(B) + 12\cdot c_n(B))\cdot y_{p,q}\nonumber\\
    &+& \sum_{a\in A} s(a)\cdot c_e(B)\cdot z_{a,p,q} \Big) \nonumber\\
\text{subject to}&&\nonumber\\
\forall_{a\in A}, 
    && \sum_{p=1}^{k} x_{a,p} \geq 1\nonumber\\
\forall_{\{a,q\}\in A\times Q},
    &&  \sum_{p=1}^{k} z_{a,p,q} \geq q(a) \nonumber\\
\forall_{\{a,p,q\}\in A\times [1..k]\times Q}, 
    && x_{a,p} - z_{a,p,q} \geq 0 \nonumber\\
\forall_{\{p,q\}\in [1..k]\times Q}, 
    &&  \sum_{a\in A} z_{a,p,q} - y_{p,q} \geq 0 \nonumber\\
\forall_{\{p,q\}\in [1..k]\times Q}, 
    &&  K\cdot y_{p,q} - \sum_{a\in A} z_{a,p,q}  \geq 0 \nonumber\\
\forall_{\{a,p,q\}\in A\times [1..k]\times Q}, 
    && z_{a,p,q} - (x_{a,p} + y_{p,q}) \geq -1 \nonumber\\
\forall_{p\in[1..k]},
    && \sum_{a\in A} x_{a,p} - u_p \geq 0 \nonumber\\
\forall_{p\in[1..k]},
    && K\cdot u_p - \sum_{a\in A} x_{a,p} \geq 0 \nonumber\\    
\sum_{p=1}^{k} \Big((16\cdot c_e(B) &+& 12 \cdot c_n(B)) \cdot u_p  \nonumber \\ 
+ \sum_{a\in A} s(a) \!\!&\cdot&\!\! c_e(B)\cdot x_{a,p}  \Big)\leq s(B)\cdot (1+\alpha)\nonumber
\end{eqnarray}
%\end{mdframed}
\caption{ILP formulation for the overlapping partitioning}
\label{fig:ov-ilp}
\end{figure}

The final ILP formulation for the overlapping partitioning is given in
Figure~\ref{fig:ov-ilp}. We have a total of $2\cdot|A|^2\cdot|Q| +
3\cdot|A|\cdot|Q| + 3\cdot|A| + 1$ constraints and the objective function
contains $|A|\cdot|Q|\cdot(1+|A|)$ variables. 

