\section{Introduction}\label{sec:introduction}

We are living in an ever more connected world, where the data generated by
people, software systems, and the physical world is more accessible than
before and is much larger in volume, variety, and velocity. In many
application domains, such as telecommunications and social media, live data
recording the interactions between people, systems, and the environment is
available for analysis. This data often takes the form of a temporally
evolving graph, where entities are the vertices and the interactions between
them are the edges. We call such graphs \emph{interaction graphs}. 

Data analytics performed on interaction graphs can bring new business insights
and improve decision making. For instance, the graph structure may represent
the interactions in a social network, where finding communities in the graph
can facilitate targeted advertising. In the telecommunications (telco) domain,
call details records (CDRs) can be used to capture the call interactions
between people, and locating closely connected groups of people can be used
for generating promotions. 

Interaction graphs are temporal in nature, and more importantly, they are
append-only. This is in contrast to relationship graphs, which are updated via
insertion and deletion operations. An example of a relationship graph is a
social network capturing the follower-followee relationship among users.
Examples of interactions graphs include CDR graphs capturing calls between
telco customers or mention graphs capturing interactions between users of a
micro-blogging service. The append-only nature of the interaction graphs make
storing them on disk a necessity. Furthermore, the analysis of this historical
interaction data form an important part of the analytical landscape.

The ability to efficiently support historical analysis over interaction graphs
require effective solutions for the problem of \emph{data layout} on disk.
Most graph algorithms are characterized by locality of
access~\cite{localityLayout}, which is a direct result of the traversal-based
nature of most of the graph algorithms. This is often taken advantage of by
co-locating edges in close proximity within the same disk
blocks~\cite{gstoreThesis}. This way, once a disk block is loaded into main
memory buffers, several edges from it can be used for processing, reducing the
disk I/O. 

In interaction graphs, the locality of access is even more pronounced. First, 
the analysis to be performed on the interactions can be restricted to a
temporal view of the graph, such as finding the influential users over a given
week of interactions. This means that edges that are temporally close are to
be accessed together. Second, traversals are again key to many graph
algorithms, such as connected components, clustering coefficient, PageRank,
etc. This means that edges that are close by in terms of the path between
their incident vertices as well as their timestamps should be located together
with the same blocks.  In our earlier work~\cite{gedik14}, we introduced an
interaction graph database that works on this principle of access locality. It
uses a disk organization that consists of as set of blocks, each containing a
list of \emph{temporal neighbor lists}. A temporal neighbor list contains a
head vertex and a set of incident edges within a time range. The layout
optimizer aims at bringing together, into the same disk block, temporal
neighbor lists that are ($i$) close in terms of their temporal ranges, ($ii$)
have many edges between them, and ($iii$) have few edges going into temporal
neighbor lists outside the block.

Many real world graph databases contain attributes. In the case of interaction
graphs, the attributes can be considered as properties associated with the
edges representing the interactions. Attributes can be stored in two ways,
either separately (e.g., in a relational table), or locally with the temporal
neighbor lists.  If they are stored separately, then the graph database cannot
take full advantage of locality optimizations performed for block
organization. The database must go back and forth between the disk blocks to
access the edge attributes.  On the other hand, if attributes are stored
locally in the disk blocks containing the graph structure, then there can be
significant overhead due to disk I/O if only a few attributes are needed to
answer a query. 

To query an interaction graph, most algorithms traverse the graph structure to
access the relevant attributes.  Frequently, there are correlations among the
attributes accessed by different queries. For example, queries $q_1$ and $q_5$
might access attributes $a_1$ and $a_2$, while queries $q_2$, $q_3$ and $q_4$
access attributes $a_3$ and $a_4$. Because interaction graphs are temporal,
the co-access correlations for the attributes can vary for different temporal
regions.  Moreover, the co-access correlations might be unknown at the
insertion time, but be discovered later, when the workload is known.

It is widely recognized that query workload and disk layout have a significant
impact on database performance~\cite{alagiannis14,grund10,stonebraker05}.  For
table-based relational databases, this fact has led database designers to
develop alternative approaches for storage layout: row-oriented
storage~\cite{rowOrg} is more efficient when queries access many attributes
from a small number of records, and column-oriented storage~\cite{colOrg} is
more efficient when queries access a small number of attributes from many
records~\cite{stonebraker05}.  Unfortunately, although interaction graph
databases, like relational databases, are the target of diverse query
workloads, there is no clear correspondence to a row-oriented or
column-oriented storage layout.

This paper presents an adaptive disk layout called the \emph{railway layout}
and associated algorithms for optimizing disk block storage for interaction
graphs. The key idea is to divide blocks into one or more sub-blocks, where
each sub-block contains a subset of the attributes (potentially overlapping),
but the entire graph structure is replicated within each sub-block. This way,
a query can be answered completely by only reading the sub-blocks that contain
the attributes of interest, reducing the overall I/O. 

There a number of challenges in achieving an effective adaptive layout. First,
we need to find the partitioning of attributes that minimized the query I/O.
To address this, we model the problems of overlapping and non-overlapping
attribute partitioning as mixed-integer linear programs (ILPs), and provide
optimal solutions that minimize the query I/O cost. Second, the query
workload, and thus the attribute access pattern can change over time. For this
purpose, our railway layout supports customization of the attribute
partitioning of sub-blocks on a per-block basis. Third, such flexibility
necessitates online configuration of attribute partitioning as the query
workload evolves, which in turn requires fast algorithms for performing the
attribute partitioning. For this purpose, we develop greedy heuristic
algorithms for both overlapping and non-overlapping partitioning scenarios.
These algorithms can scale to larger number of attributes, yet provide close
to optimal query I/O performance. Finally, the railroad layout trades off
storage space to gain improved query I/O performance. The storage overhead is
more pronounced for the case of overlapping partitioning. To address this, we
limit the amount of storage  overhead that can be tolerated, and integrate
this limit to both our ILP formulations, as well as our greedy heuristics.

In summary, this paper makes the following contributions:
\begin{itemize}
\item We introduce the railroad layout for adaptive organization of interaction
graphs on disk. 
\item We introduce optimal ILP formulations for partitioning disk blocks into 
sub-blocks with overlapping and non-overlapping attributes, given a query
workload. Our formulation also support upper bounding the amount of storage
overhead introduced as a result of the railroad layout.
\item To support online adaptation, we develop greedy heuristics that can scale
better compared to the ILP alternatives, yet achieve close to optimal query
I/O.
\item We provide an extensive experimental study comparing our approach to a
few baseline alternatives.
\end{itemize}

The rest of the paper is organized as follows. Section...

% This points to the need for
% adaptively optimizing the layout (somewhat similar to H2O \cite{alagiannis14}
% and HYRISE~\cite{grund10}).%



