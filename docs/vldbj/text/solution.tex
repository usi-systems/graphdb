\clearpage
\newpage
\section{Partitioning}
This section is still a work-in-progress. We are trying to find the algorithms.

\subsection{Non-Overlapping Attributes}

\paragraph*{Problem.$\,$} \emph{Find a true partitioning of attributes that
minimizes the query I/O and bounds the storage cost by some upper limit.}

\subsubsection{Integer Linear Program Formulation}\label{subsubsec:nov-ilp}
We present an ILP formuation of the problem. For this purpose, we define a 
number of binary ($0$ or $1$) variables: 
\begin{itemize}
\item $x_{a,p}$: $1$ if attribute $a$ is in partition $p$, $0$ otherwise.
\item $y_{p,q}$: $1$ if partition $p$ is used by query $q$, $0$ otherwise.
\item $z_{a,p,q}$: $1$ if partition $p$ is used by query $q$ and attribute $a$
is in partition $p$, $0$ otherwise.
\item $u_{p}$: $1$ if partition $p$ is assigned at least $1$ attribute, $0$ otherwise.
\end{itemize}

Each of these variables serve a purpose:
\begin{itemize}
\item $x$s define the attribute-to-partition assignment.
\item $y$s help formulate the query I/O contribution
of partitions, excluding their assigned attributes.
\item $z$s help formulate the query I/O contribution
of partitions, only considering their assigned attributes.
\item $u$s help formulate the storage overhead requirement.
\end{itemize}

We also define a helper notation for representing whether a variable is
accessed by a query or not: $q(a)\equiv \mathbf{1}(a \in q.A)$, where
$\mathbf{1}$ is the indicator function. We assume that there are a maximum of
$k$ partitions. $k$ can be taken as $|A|$, as there cannot be more partitions
than there are attributes.

We are now ready to state the ILP formulation. We start with the objective
function, that is the total query I/O:
\begin{eqnarray}
\sum_{q\in Q} w(q) \cdot \Big(\sum_{p=1}^{k} \!\!&&\!\! (16\cdot c_e(B) + 12\cdot c_n(B))\cdot
y_{p,q}\nonumber\\ 
&+& \sum_{a\in A} s(a)\cdot c_e(B)\cdot z_{a,p,q}\Big)\label{eq:no-obj}
\end{eqnarray}

In Eq.~\ref{eq:no-obj}, we simply sum for each query and each partition, and
add the I/O cost of reading in the structural information found in a
sub-block, if the partition is used by the query. We then sum over each
attribute as well, and add the I/O cost of reading in the attributes. Note
that $z_{a,p,q}$ could have been  replaced with $x_{a,p}\cdot y_{p,q}$, but
that would make the objective function non-linear. 

We are now ready to state our constraints. Our first constraint is that, each
attribute must be assigned to a single partition. Formally:
\begin{eqnarray}
\forall_{a\in A}, \sum_{p=1}^{k} x_{a,p} = 1
\end{eqnarray}

Our second constraint is that, if a query $q$ contains an attribute $a$
assigned to a partition $p$, then partition $p$ is used by the query, i.e.,
$y_{p,q}=1$. In essence, we want to state: $\forall_{\{p,q\}\in [1..k]\times
Q}, y_{p,q} = \mathbf{1}(\sum_{a\in A} q(a)\cdot x_{a,p}>0)$. 

In order to formulate this constraint, we use the following ILP  construction:
Assume we have two variables, $\beta_1$ and $\beta_2$, where $\beta_2\in[0,1]$
and $\beta_1\geq 0$. We want to implement the following constraint: $\beta_2 =
\mathbf{1}(\beta_1 > 0)$. This could be expressed as a linear constraint as
follows, where $K$ is a large constant guaranteed to be larger than $\beta_1$
for practical purposes:
\begin{eqnarray}
&& \beta_1 - \beta_2 \geq 0\nonumber\\
&& K\cdot\beta_2 - \beta_1 \geq 0\label{eq:beta-ilp}
\end{eqnarray}

We now apply this construction to our second constraint, where
$\beta_1=\sum_{a\in A} q(a)\cdot x_{a,p}$ and $\beta_2=y_{p,q}$. This results
in the following linear constraints:
\begin{eqnarray}
\forall_{\{p,q\}\in [1..k]\times Q}, 
    &&  \sum_{a\in A} q(a)\cdot x_{a,p} - y_{p,q} \geq 0 \nonumber\\
\forall_{\{p,q\}\in [1..k]\times Q}, 
    &&  K\cdot y_{p,q} - \sum_{a\in A} q(a)\cdot x_{a,p}  \geq 0 
\end{eqnarray}

Our third constraint is that, if an attribute $a$ is assigned to a partition
$p$, and partition $p$ is used by a query $q$, then the corresponding $z$
variable must be set to $1$. That is, we want: $\forall_{\{a,p,q\}\in A\times
[1..k]\times Q}, z_{a,p,q}=\mathbf{1}(x_{a,p} = y_{p,q} = 1)$. We express this
as a linear  constraint, as follows:
\begin{eqnarray}
\forall_{\{a,p,q\}\in A\times [1..k]\times Q},
    && z_{a,p,q} - (x_{a,p} + y_{p,q}) \geq -1\label{eq:no-z}
\end{eqnarray}

In Eq.~\ref{eq:no-z}, when the $x$ and $y$ variables are both $1$, the  $z$
variable is simply forced to be $1$. Otherwise, the $z$ variable can be either
$0$ or $1$, but since the $z$ variables appear in the objective fucntion as
positive terms, the solver will set them to $0$, which is what we want. 

Our fourth constraint is that, if a partition is non-empty, then its
corresponding $u$ variable must be set to $0$. In other words,  we want
$\forall_{p\in[1..k]}, u_p = \mathbf{1}(\sum_{a\in A} x_{a,p}>0)$. This is
expressed as linear constraints, as follows:
\begin{eqnarray}
\forall_{p\in[1..k]},
    && \sum_{a\in A} x_{a,p} - u_p \geq 0 \nonumber\\
\forall_{p\in[1..k]},
    && K\cdot u_p - \sum_{a\in A} x_{a,p} \geq 0 \label{eq:no-u}
\end{eqnarray}

Eq.~\ref{eq:no-u} uses the same construction as the second constraint, where
$\beta_1=\sum_{a\in A} x_{a,p}$ and $\beta_2=u_p$.

Our fifth, and the last, constraint deals with the storage overhead. We want to
 make sure that the storage overhead does not go over $\alpha$. The storage
overhead depends on the number of partitions used. That means that the only 
ILP variables it depends on is the $u$s. In particular, the number of
partitions used is given by $\sum_{p=1}^{k} u_p$. This results in the
following linear constraint:
\begin{equation}
\sum_{p=1}^{k} u_p \leq 1 + \frac{\alpha}
  {1-\frac{c_e(B)\cdot \sum_{a\in A} s(a)}{s(B)}}
\end{equation}

\begin{figure}[!t]
\begin{mdframed}
\begin{eqnarray}
\text{minimize}  
    \sum_{q\in Q} w(q)\cdot \Big(\sum_{p=1}^{k} \!\!&&\!\! (16\cdot c_e(B) + 12\cdot c_n(B))\cdot y_{p,q}\nonumber\\
    &+& \sum_{a\in A} s(a)\cdot c_e(B)\cdot z_{a,p,q} \Big) \nonumber\\
\text{subject to}&&\nonumber\\
\forall_{a\in A}, 
    && \sum_{p=1}^{k} x_{a,p} = 1\nonumber\\
\forall_{\{p,q\}\in [1..k]\times Q}, 
    &&  \sum_{a\in A} q(a)\cdot x_{a,p} - y_{p,q} \geq 0 \nonumber\\
\forall_{\{p,q\}\in [1..k]\times Q}, 
    &&  K\cdot y_{p,q} - \sum_{a\in A} q(a)\cdot x_{a,p}  \geq 0 \nonumber\\
\forall_{\{a,p,q\}\in A\times [1..k]\times Q},
    && z_{a,p,q} - (x_{a,p} + y_{p,q}) \geq -1\nonumber\\
\forall_{p\in[1..k]},
    && \sum_{a\in A} x_{a,p} - u_p \geq 0 \nonumber\\
\forall_{p\in[1..k]},
    && K\cdot u_p - \sum_{a\in A} x_{a,p} \geq 0 \nonumber\\    
&& \sum_{p=1}^{k} u_p \leq 1 + \frac{\alpha}
  {1-\frac{c_e(B)\cdot \sum_{a\in A} s(a)}{s(B)}} \nonumber
\end{eqnarray}
\end{mdframed}
\caption{ILP formulation for the non-overlapping partitioning}
\label{fig:nov-ilp}
\end{figure}

The final ILP formulation for the non-overlapping partitioning is given in
Figure~\ref{fig:nov-ilp}.

\subsubsection{Heuristic Solution}

\paragraph*{Approach.$\,$}
TODO

\begin{algorithm}[ht]
\scriptsize
\caption{Algorithm for partitioning blocks into sub-blocks with non-overlapping attributes.}
\label{alg:non-overlappingP}
\KwData{$B$: block, $Q$: set of queries}
$c^*\leftarrow \infty$ \tcp*{Lowest cost over all \# of partitions}
\For(\tcp*[f]{For each possible \# of partitions}){$k=1$ to $|A|$}{
   $R[i]\leftarrow \emptyset, \forall i\in [1..k]$ \tcp*{Initialize partitions}
   \For(\tcp*[f]{For each attribute}){$a \in A$\textnormal{, in decr.\/ order of }$f(a)$}{
      $c\leftarrow \infty$ \tcp*{Lowest cost over all assignments}  
      $j\leftarrow -1$ \tcp*{Best partition assignment}
      \For(\tcp*[f]{For each partition assignment}){$i\in [1..k]$} {
         $R[i]\leftarrow R[i] \cup \{a\}$\tcp*{Assign attribute}
         \If(\tcp*[f]{If query cost is lower}){$L(R, B, Q)<c$}{
            $c\leftarrow L(R, B, Q)$\tcp*{Update the lowest cost}
            $j\leftarrow i$\tcp*{Update the best partition}
         }
         $R[i]\leftarrow R[i] \setminus \{a\}$\tcp*{Un-assign attribute}
      }
      $R[j]\leftarrow R[j] \cup \{a\}$\tcp*{Assign to best partition}
   }
   \lIf(\tcp*[f]{If solution infeasible}){$H(R, B, Q)>\alpha$}{\textbf{break}}
   \If(\tcp*[f]{If solution has lower cost}){$L(R, B, Q)<c^*$}{
     $c^* \leftarrow L(R, B, Q)$\tcp*{Update the lowest cost}
     $\mathcal{P}(B)\leftarrow R$\tcp*{Update the best partitioning}
   } 
}
\Return $\mathcal{P}(B)$ \tcp*{Final set of sub-blocks}
\end{algorithm} 

\clearpage
\newpage
\subsection{Overlapping Attributes}

\paragraph*{Problem.$\,$} \emph{Find an overlapping partitioning of attributes
that minimizes the query I/O and bounds the storage cost by some upper
limit.}

\subsubsection{Integer Linear Program Formulation}
We present an ILP formuation of the problem, as we did for the case of
non-overlapping partitions in Section~\ref{subsubsec:nov-ilp}. We use the same
set of variables and the same objective function. However, the formulation of
the constraints differ. 

Our first constraint is that, each attribute must be assigned to at leat one
partition. Formally:
\begin{eqnarray}
\forall_{a\in A}, \sum_{p=1}^{k} x_{a,p} \geq 1
\end{eqnarray}

As our second constraint, we require that for each attribute used by a query,
there needs to be a partition that is used by that query and that contains the
attribute in question. Formally:
\begin{eqnarray}
\forall_{\{a,q\}\in A\times Q}, \sum_{p=1}^{k} z_{a,p,q} \geq q(a) 
\end{eqnarray}

\begin{sloppypar}
As our third constraint, we require that if a query is using an attribute from
a partition, then that partition must contain the attribute. I.e., we need
to link the $z$ variables with the $x$ variables as
$\forall_{\{a,p,q\}\in A\times [1..k]\times Q}, (z_{a,p,q} = 1) \implies 
(x_{a,p} = 1)$. This can be state as linear constraints:
\begin{eqnarray}
\forall_{\{a,p,q\}\in A\times [1..k]\times Q}, x_{a,p} - z_{a,p,q} \geq 0 
\end{eqnarray}
\end{sloppypar}

As our fourth constraint, we require that if a query is using at least one
attribute from a partition, then that partition must be used by the query.
I.e., we need to link the $z$ variables with the $y$ variables as
$\forall_{\{p,q\}\in [1..k]\times Q}, y_{p,q} = \mathbf{1}(\sum_{a\in A}
z_{a,p,q}>0)$. As before, we use the ILP construction from
Eq.~\ref{eq:beta-ilp} for this, where $\beta_2=y_{p,q}$ and $\beta_1 =
\sum_{a\in A} z_{a,p,q}$. We get:
\begin{eqnarray}
\forall_{\{p,q\}\in [1..k]\times Q}, 
    &&  \sum_{a\in A} z_{a,p,q} - y_{p,q} \geq 0 \nonumber\\
\forall_{\{p,q\}\in [1..k]\times Q}, 
    &&  K\cdot y_{p,q} - \sum_{a\in A} z_{a,p,q} \geq 0 
\end{eqnarray}

Our fifth constraint is that, if an attribute $a$ is assigned to a partition
$p$, and partition $p$ is used by a query $q$, then the corresponding $z$
variable must be set to $1$. This is same as the formulation for the
non-overlapping case from Eq.~\ref{eq:no-z}.

Our sixth constraint is that, if a partition is non-empty, then its
corresponding $u$ variable must be set to $0$. Again, this is same as the
formulation for the non-overlapping case from Eq.~\ref{eq:no-u}.

Our seventh, and the last, constraint deals with the storage overhead. However,
the storage overhead formulation for the overlapping case is different from 
the one for the non-overlapping. This is because the overhead does not merely
depend on the number of partitions, as attributes might have to be loaded
multiple times from different partitions. As a result, we express the overhead
using base variables as in the objective function. Formally:
\begin{eqnarray}
\sum_{p=1}^{k} \Big((16\cdot c_e(B) &+& 12 \cdot c_n(B)) \cdot u_p  \nonumber \\ 
+ \sum_{a\in A} s(a) \!\!&\cdot&\!\! c_n(B)\cdot x_{a,p} \Big) \leq s(B)\cdot (1+\alpha)
\end{eqnarray}

\begin{figure}[!t]
\begin{mdframed}
\begin{eqnarray}
\text{minimize}  
    \sum_{q\in Q} w(q)\cdot \Big(\sum_{p=1}^{k} \!\!&&\!\! (16\cdot c_e(B) + 12\cdot c_n(B))\cdot y_{p,q}\nonumber\\
    &+& \sum_{a\in A} s(a)\cdot c_e(B)\cdot z_{a,p,q} \Big) \nonumber\\
\text{subject to}&&\nonumber\\
\forall_{a\in A}, 
    && \sum_{p=1}^{k} x_{a,p} \geq 1\nonumber\\
\forall_{\{a,q\}\in A\times Q},
    &&  \sum_{p=1}^{k} z_{a,p,q} \geq q(a) \nonumber\\
\forall_{\{a,p,q\}\in A\times [1..k]\times Q}, 
    && x_{a,p} - z_{a,p,q} \geq 0 \nonumber\\
\forall_{\{p,q\}\in [1..k]\times Q}, 
    &&  \sum_{a\in A} z_{a,p,q} - y_{p,q} \geq 0 \nonumber\\
\forall_{\{p,q\}\in [1..k]\times Q}, 
    &&  K\cdot y_{p,q} - \sum_{a\in A} z_{a,p,q}  \geq 0 \nonumber\\
\forall_{\{a,p,q\}\in A\times [1..k]\times Q}, 
    && z_{a,p,q} - (x_{a,p} + y_{p,q}) \geq -1 \nonumber\\
\forall_{p\in[1..k]},
    && \sum_{a\in A} x_{a,p} - u_p \geq 0 \nonumber\\
\forall_{p\in[1..k]},
    && K\cdot u_p - \sum_{a\in A} x_{a,p} \geq 0 \nonumber\\    
\sum_{p=1}^{k} \Big((16\cdot c_e(B) &+& 12 \cdot c_n(B)) \cdot u_p  \nonumber \\ 
+ \sum_{a\in A} s(a) \!\!&\cdot&\!\! c_n(B)\cdot x_{a,p}  \Big)\leq s(B)\cdot (1+\alpha)\nonumber
\end{eqnarray}
\end{mdframed}
\caption{ILP formulation for the overlapping partitioning}
\label{fig:ov-ilp}
\end{figure}

The final ILP formulation for the overlapping partitioning is given in
Figure~\ref{fig:ov-ilp}.

\subsubsection{Heuristic Solution}

\paragraph*{Approach.$\,$} We start the algorithm with a   partitioning based
on what queries we have seen. Every query gets its own sub-block. This is the
``ideal'' partitioning, because the I/O cost would be minimized for every
query that we would have seen. The algorithm iteratively combines the two
partitions that are closest together.  After each combination of partitions,
the algorithm calculate the storage overhead for the partitioning. The
algorithm stops when the  storage cost is below some specified threshold.  The
result is the block partitioning.

\begin{algorithm}[ht]
\scriptsize
\caption{Algorithm for partitioning blocks into sub-blocks with overlapping attributes.}
\label{alg:overlappingP}
\KwData{$B$: block, $Q$: set of queries}
$\mathcal{P}(B) \leftarrow \{q.A: q\in Q\}$ \tcp*{Each query gets its own sub-block}
\While(\tcp*[f]{Until storage overhead is below $\alpha$}){$H(\mathcal{P},B) > \alpha$}{
  $c^{*}\gets \infty $ \tcp*{Lowest cost over all sub-block pairs}
  $(b_x,b_y)\gets (\emptyset,\emptyset)$ \tcp*{Sub-block pair with the lowest cost}
  \For(\tcp*[f]{For each pair of blocks}){$\{b_i,b_j\}\in\mathcal{P}(B)$}{
    $\mathcal{P'}(B) \leftarrow \mathcal{P}(B) \setminus \{b_i, b_j\} \cup \{b_i \cup b_j\}$\\
    $c\gets \frac{L(\mathcal{P}',B,Q)-L(\mathcal{P},B,Q)}{H(\mathcal{P},B)-H(\mathcal{P}',B)}$\tcp*{Cost of merge}
    \If(\tcp*[f]{Cost is lower}){$c<c^{*}$}{
        $c^{*}\gets c$\tcp*{Update the lowest cost}
        $(b_x,b_y)\gets (b_i,b_j)$\tcp*{Update the best pair}
    }
  }
  $\mathcal{P}(B) \leftarrow \mathcal{P}(B) \setminus \{b_x, b_y\} \cup \{b_x \cup b_y\}$ \\
}
\Return $ \mathcal{P}(B)$  \tcp*{Final set of sub-blocks}
\end{algorithm} 

